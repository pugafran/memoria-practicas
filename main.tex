%%%%%%%%%%%%%%%%%%%%%%%%%%%%%%%%%%%%%%%%%
% Masters/Doctoral Thesis 
% LaTeX Template
% Version 2.5 (27/8/17)
%
% This template was downloaded from:
% http://www.LaTeXTemplates.com
%
% Version 2.x major modifications by:
% Vel (vel@latextemplates.com)
%
% This template is based on a template by:
% Steve Gunn (http://users.ecs.soton.ac.uk/srg/softwaretools/document/templates/)
% Sunil Patel (http://www.sunilpatel.co.uk/thesis-template/)
%
% Template license:
% CC BY-NC-SA 3.0 (http://creativecommons.org/licenses/by-nc-sa/3.0/)
%
% // ========= //
%
% This template has been partially modified by Alejandro Rodríguez López
%
%%%%%%%%%%%%%%%%%%%%%%%%%%%%%%%%%%%%%%%%%

%----------------------------------------------------------------------------------------
%	PACKAGES AND OTHER DOCUMENT CONFIGURATIONS
%----------------------------------------------------------------------------------------

\documentclass[
11pt, % The default document font size, options: 10pt, 11pt, 12pt
oneside, % Two side (alternating margins) for binding by default, uncomment to switch to one side
spanish, % TODO: Set language [english, spanish, ngerman]
singlespacing, % Single line spacing, alternatives: onehalfspacing or doublespacing
%draft, % Uncomment to enable draft mode (no pictures, no links, overfull hboxes indicated)
%nolistspacing, % If the document is onehalfspacing or doublespacing, uncomment this to set spacing in lists to single
%liststotoc, % Uncomment to add the list of figures/tables/etc to the table of contents
%toctotoc, % Uncomment to add the main table of contents to the table of contents
%parskip, % Uncomment to add space between paragraphs
%nohyperref, % Uncomment to not load the hyperref package
headsepline, % Uncomment to get a line under the header
%chapterinoneline, % Uncomment to place the chapter title next to the number on one line
%consistentlayout, % Uncomment to change the layout of the declaration, abstract and acknowledgements pages to match the default layout
]{MastersDoctoralThesis} % The class file specifying the document structure

\usepackage[utf8]{inputenc} % Required for inputting international characters
\usepackage[T1]{fontenc} % Output font encoding for international characters
\usepackage{pgffor}
\usepackage{listings}
\usepackage{hyperref}
\usepackage{xcolor}
\usepackage{tabularx}
\usepackage{listings-rust}
\usepackage{tabularx, multirow}
\usepackage{amsmath}

\colorlet{punct}{red!60!black}
\definecolor{background}{HTML}{EEEEEE}
\definecolor{delim}{RGB}{20,105,176}
\colorlet{numb}{magenta!60!black}

\lstdefinelanguage{json}{
    basicstyle=\normalfont\ttfamily,
    numbers=left,
    numberstyle=\scriptsize,
    stepnumber=1,
    numbersep=8pt,
    showstringspaces=false,
    breaklines=true,
    frame=lines,
    backgroundcolor=\color{background},
    literate=
     *{0}{{{\color{numb}0}}}{1}
      {1}{{{\color{numb}1}}}{1}
      {2}{{{\color{numb}2}}}{1}
      {3}{{{\color{numb}3}}}{1}
      {4}{{{\color{numb}4}}}{1}
      {5}{{{\color{numb}5}}}{1}
      {6}{{{\color{numb}6}}}{1}
      {7}{{{\color{numb}7}}}{1}
      {8}{{{\color{numb}8}}}{1}
}

\definecolor{lightgray}{rgb}{.9,.9,.9}
\definecolor{darkgray}{rgb}{.4,.4,.4}
\definecolor{purple}{rgb}{0.65, 0.12, 0.82}

\lstdefinelanguage{JavaScript}{
  keywords={const, let, var, typeof, new, true, false, catch, function, return, null, catch, switch, var, if, in, while, do, else, case, break},
  keywordstyle=\color{blue}\bfseries,
  ndkeywords={class, export, boolean, throw, implements, import, this},
  ndkeywordstyle=\color{darkgray}\bfseries,
  identifierstyle=\color{black},
  sensitive=false,
  comment=[l]{//},
  morecomment=[s]{/*}{*/},
  commentstyle=\color{purple}\ttfamily,
  stringstyle=\color{red}\ttfamily,
  morestring=[b]',
  morestring=[b]"
}

\lstset{
   backgroundcolor=\color{lightgray},
   extendedchars=true,
   basicstyle=\footnotesize\ttfamily,
   showstringspaces=false,
   showspaces=false,
   numbers=left,
   numberstyle=\footnotesize,
   numbersep=9pt,
   tabsize=2,
   breaklines=true,
   showtabs=false,
   captionpos=b
}

\usepackage{enumitem}
\usepackage{tikz-uml}
% Remove the problematic code
\usepackage{float}

\usepackage{tcolorbox}
\newtcolorbox{notebox}{
  colback=white,
  colframe=black,
  boxrule=1pt,
  arc=4pt,
  fonttitle=\bfseries,
  title=NOTA
}

\newtcolorbox{examplebox}{
  colback=white,
  colframe=blue,
  boxrule=1pt,
  arc=4pt,
  fonttitle=\bfseries,
  title=EJEMPLO
}

\usepackage{mathpazo} % Use the Palatino font by default
%\usepackage[backend=bibtex,style=authoryear,natbib=true,language=english]{biblatex} % Use the bibtex backend with the authoryear citation style (which resembles APA)

%\addbibresource{bibliography.bib} % The filename of the bibliography

\usepackage[autostyle=true]{csquotes} % Required to generate language-dependent quotes in the bibliography
\usepackage{titlesec}

\newcommand{\bold}[1]{\textbf{#1}\ }
\newcommand{\italic}[1]{\textit{#1}\ }
\newcommand{\frontend}{\bold{frontend}}
\newcommand{\backend}{\bold{backend}}
\newcommand{\ReactJS}{\href{https://react.dev/}{\bold{ReactJS}}}
\newcommand{\Flask}{\href{https://flask.palletsprojects.com/en/3.0.x/}{\bold{Flask}}}
\newcommand{\SQLAlchemy}{\href{https://www.sqlalchemy.org/}{\bold{SQL Alchemy}}}
\newcommand{\TypeScript}{\href{https://www.typescriptlang.org/}{\bold{TypeScript}}}
\newcommand{\JavaScript}{\href{https://www.javascript.com/}{\bold{JavaScript}}}
\newcommand{\Python}{\href{https://www.python.org/}{\bold{Python}}}
\newcommand{\Java}{\href{https://www.java.com/en/}{\bold{Java}}}
\newcommand{\Rust}{\href{https://www.rust-lang.org/}{\bold{Rust}}}
\newcommand{\C}{\href{https://en.wikipedia.org/wiki/C_(programming_language)}{\bold{C}}}
\newcommand{\Haskell}{\href{https://www.haskell.org/}{\bold{Haskell}}}

\newcommand\Chapter[2]{
  \chapter[#1: {\itshape#2}]{#1\\\hfill\Large\itshape#2}
  \label{chap:#1}
}

\newcommand\Section[1]{\section{#1}\label{sec:#1}}
\newcommand\Subsection[1]{\subsection{#1}\label{ssec:#1}}
\newcommand\Subsubsection[1]{\subsubsection{#1}\label{sssec:#1}}
\newcommand\Paragraph[1]{\paragraph{#1}\label{par:#1}}

\newcommand\uml[2]{
  \begin{figure}[h]
    \begin{center}
      #1
    \end{center}
    \caption{#2}
  \end{figure}
}

%----------------------------------------------------------------------------------------
%	PARAGRAPH SETTINGS
%----------------------------------------------------------------------------------------

\setlength{\parskip}{\baselineskip}%

%----------------------------------------------------------------------------------------
%	MARGIN SETTINGS
%----------------------------------------------------------------------------------------

\geometry{
	paper=a4paper, % Change to letterpaper for US letter
	inner=2.5cm, % Inner margin
	outer=3.8cm, % Outer margin
	bindingoffset=.5cm, % Binding offset
	top=1.5cm, % Top margin
	bottom=1.5cm, % Bottom margin
	%showframe, % Uncomment to show how the type block is set on the page
}

%----------------------------------------------------------------------------------------
%	THESIS INFORMATION
%----------------------------------------------------------------------------------------

% Should you want to have an href alongside the name, state it inside the variable using:
% \href{url}{name}

% TODO: Your thesis title, print it with \ttitle
\thesistitle{Copilot inteligente para consultas LINQ/SQL}
% TODO: Your tutor name, print it with \tutor
\tutor{Moldón Redondo, Daniel} 
% TODO: Your other tutor name, print it with \tutorBis
\tutorBis{Costa Cortez, Nahuel Alejandro} 
% TODO: Your degree name, print it with \degreename
\degree{la asignatura Prácticas de Empresa del grado Ingeniería Informática en Tecnologías de la Información} 
% TODO: Your name, print it with \authorname
\author{Puga Lojo, Francisco Gabriel} 
% TODO: Your peer name, print it with \authornameBis
\authorBis{AutorBis} 
% TODO: Your address, print it with \addressname
\addresses{Address} 

% TODO: Your subject area, print it with \subjectname
\subject{} 
% TODO: Keywords for your thesis, print it with \keywordnames
\keywords{Keywords} 
% TODO: Your university's name and URL, print it with \univname
\university{UNIVERSIDAD DE OVIEDO} 
% TODO: Your department's name and URL, print it with \deptname
\department{Ingeniería Informática en Tecnologías de la Información} 
% TODO: Your research group's name and URL, print it with \groupname
\group{Group} 
% TODO: Your faculty's name and URL, print it with \facname
\faculty{Escuela Politécnica de Ingeniería de Gijón (EPI)} 

\AtBeginDocument{
	\hypersetup{pdftitle=\ttitle} % Set the PDF's title to your title
	\hypersetup{pdfauthor=\authorname} % Set the PDF's author to your name
	\hypersetup{pdfkeywords=\keywordnames} % Set the PDF's keywords to your keywords
}

\begin{document}

\frontmatter % Use roman page numbering style (i, ii, iii, iv...) for the pre-content pages

\pagestyle{plain} % Default to the plain heading style until the thesis style is called for the body content

%----------------------------------------------------------------------------------------
%	TITLE PAGE
%----------------------------------------------------------------------------------------

\begin{titlepage}
\begin{center}

\vspace*{.06\textheight}
{\scshape\LARGE \univname\\\facname\par}\vspace{1.5cm} % University name
\textsc{\Large \subjectname}\\[0.5cm] % Thesis type

\HRule \\[0.4cm] % Horizontal line
{\huge \bfseries \ttitle\par}\vspace{0.4cm} % Thesis title
\HRule \\[1.5cm] % Horizontal line
 
\begin{minipage}[t]{\textwidth}
	\begin{flushleft} \large
		\emph{Autor:}\\
		\authorname\\
    \vspace{2em}
	\end{flushleft}
\end{minipage}
\begin{minipage}[t]{\textwidth}
	\begin{flushleft} \large
    \vspace{2em}
	\end{flushleft}
\end{minipage}
\begin{minipage}[t]{0.8\textwidth}
	\begin{flushright} \large
		\emph{Tutor (Mecalux):} \\
		\tutorName
    \vspace{2em}
	\end{flushright}
\end{minipage}\\
\begin{minipage}[t]{0.8\textwidth}
	\begin{flushright} \large
		\emph{Tutor (EPI):} \\
		\tutorBisName
	\end{flushright}
\end{minipage}\\[3cm]
 
\vfill

% University requirement text
\large\textit{\requirements}\\[0.3cm] 
% \textit{en}\\[0.4cm]
% \groupname\\[2cm] % Research group name and department name
 
\vfill

{\large \today}\\[4cm] % Date
%\includegraphics{Logo} % University/department logo - uncomment to place it
 
\vfill
\end{center}
\end{titlepage}

%----------------------------------------------------------------------------------------
%	DECLARATION PAGE
%----------------------------------------------------------------------------------------

%\begin{declaration}
%\addchaptertocentry{\authorshipname} % Add the declaration to the table of contents
%\noindent Yo, \authorname, declaro que este documento titulado, \enquote{\ttitle} y el trabajo presentado en él son de mi propiedad.
%Afirmo que:

%\begin{itemize} 
	%\item Este trabajo fue realizado completa o parcialmente durante mi estancia como estudiante en el grado de \`degree'name.
	%\item Aquellas partes de este documento que hayan sido previamente publicadas se encuentran debidamente indicadas.
	%\item Aquellas partes de este documento que se apoyen en trabajos previamente publicados se encuentran debidamente indicadas.
	%\item Todas las fuentes utilizadas para la realizacion de este trabajo se encuentran listadas.
%\end{itemize}

%\vfill

%\noindent Signed: \authorname\\
%\rule[0.5em]{25em}{0.5pt} % This prints a line for the signature
 
%\noindent Date: Julio 13, 2023\\
%\rule[0.5em]{25em}{0.5pt} % This prints a line to write the date
%\end{declaration}

%\cleardoublepage

%----------------------------------------------------------------------------------------
%	QUOTATION PAGE
%----------------------------------------------------------------------------------------

%\vspace*{0.2\textheight}

%\noindent\enquote{
	%\itshape Thanks to my solid academic training, today I can write hundreds of words on virtually any topic
	%without possessing a shred of information, which is how I got a good job in journalism.
%}\bigbreak

%\hfill Dave Barry

%----------------------------------------------------------------------------------------
%	ABSTRACT PAGE
%----------------------------------------------------------------------------------------

\begin{abstract}
% TODO: Add the abstract

El uso de bases de datos es fundamental en el campo de la ingeniería informática. Analizar información valiosa es vital para tomar decisiones, ya que permite comparar diversas opciones de manera objetiva.

Lamentablemente, trabajar con bases de datos no es sencillo y requiere una formación que las personas fuera del ámbito de la informática generalmente no están dispuestas a adquirir.

En Mecalux, he desarrollado un asistente que facilita la generación de código LINQ SQL y proporciona explicaciones de las consultas.

El propósito de este asistente es simplificar el trabajo de los empleados de la empresa, permitiéndoles crear y comprender consultas SQL sin necesidad de conocimientos técnicos profundos, lo que optimiza los procesos y mejora la eficiencia en la toma de decisiones.
\addchaptertocentry{\abstractname}

\end{abstract}

%----------------------------------------------------------------------------------------
%	ACKNOWLEDGEMENTS
%----------------------------------------------------------------------------------------

%\begin{acknowledgements}
%\addchaptertocentry{\acknowledgementname} % Add the acknowledgements to the table of contents
%The acknowledgments and the people to thank go here, don't forget to include your project advisor\ldots
%\begin{itemize}
	%\item \groupname
		%\begin{itemize}
			%\item \tutorEmpName
			%\item Antiñolo, David Espejo
			%\item Omen, Juan David
			%\item Rico, Luis Rodríguez
		%\end{itemize}
	%\item \facname
		%\begin{itemize}
			%\item \tutorEpiName
		%\end{itemize}
%\end{itemize}
%\end{acknowledgements}

%----------------------------------------------------------------------------------------
%	LIST OF CONTENTS/FIGURES/TABLES PAGES
%----------------------------------------------------------------------------------------

\tableofcontents % Prints the main table of contents

\listoffigures % Prints the list of figures

\listoftables % Prints the list of tables

%----------------------------------------------------------------------------------------
%	ABBREVIATIONS
%----------------------------------------------------------------------------------------

\begin{abbreviations}{ll} % Include a list of abbreviations (a table of two columns)

% TODO: Add Abbreviations

%\textbf{LAH} & \textbf{L}ist \textbf{A}bbreviations \textbf{H}ere\\
%\textbf{WSF} & \textbf{W}hat (it) \textbf{S}tands \textbf{F}or\\
%\textbf{HR} & \textbf{H}uman \textbf{R}esources\\
%\textbf{BI} & \textbf{B}ase \textbf{I}mponible\\
%\textbf{CP} & \textbf{C}aso de \textbf{P}rueba\\
%\textbf{E} & Número de \textbf{E}mpleadores\\
%\textbf{I} & \textbf{I}mpuestos calculados a partir de BI\\
%\textbf{P} & Cantidad pagada en \textbf{P}réstamos por hipotecas de vivienda habitual comprada antes de 2013\\
%\textbf{R} & \textbf{R}etenciones\\
%\textbf{SP} & \textbf{S}ituación de \textbf{P}rueba\\

\end{abbreviations}

%----------------------------------------------------------------------------------------
%	PHYSICAL CONSTANTS/OTHER DEFINITIONS
%----------------------------------------------------------------------------------------

%\begin{constants}{lr@{${}={}$}l} % The list of physical constants is a three column table

% The \SI{}{} command is provided by the siunitx package, see its documentation for instructions on how to use it

%Speed of Light & $c_{0}$ & \SI{2.99792458e8}{\meter\per\second} (exact)\\
%Constant Name & $Symbol$ & $Constant Value$ with units\\

%\end{constants}

%----------------------------------------------------------------------------------------
%	SYMBOLS
%----------------------------------------------------------------------------------------

%\begin{symbols}{lll} % Include a list of Symbols (a three column table)

%$a$ & distance & \si{\meter} \\
%$P$ & power & \si{\watt} (\si{\joule\per\second}) \\
%Symbol & Name & Unit \\

%\addlinespace % Gap to separate the Roman symbols from the Greek

%$\omega$ & angular frequency & \si{\radian} \\

%\end{symbols}

%----------------------------------------------------------------------------------------
%	DEDICATION
%----------------------------------------------------------------------------------------

%\dedicatory{For/Dedicated to/To my\ldots} 

%----------------------------------------------------------------------------------------
%	BLANK PAGE
%----------------------------------------------------------------------------------------

%\blankpage

%----------------------------------------------------------------------------------------
%	THESIS CONTENT - CHAPTERS
%----------------------------------------------------------------------------------------

\mainmatter % Begin numeric (1,2,3...) page numbering

\pagestyle{thesis} % Return the page headers back to the "thesis" style

% Include the chapters of the thesis as separate files from the Chapters folder
% Uncomment the lines as you write the chapters

%% Chapter 1

\Chapter{Introducción}{El alumno y la empresa}

\href{https://www.mecalux.es/}{\bold{Mecalux}} es una empresa reconocida internacionalmente en el sector de soluciones de almacenamiento. Desde su fundación en 1966, ha desarrollado un amplio portafolio que abarca una variedad de sistemas de almacenamiento, como estanterías, almacenes automatizados y soluciones de software para la gestión de almacenes. 

Con una presencia global, Mecalux opera en numerosos países, gestionando un gran volumen de operaciones y personal especializado. Las prácticas fueron realizadas de manera presencial en las \href{https://maps.app.goo.gl/bJKvSNHAo5t1j4BZ6}{\bold{oficinas de MSS en Gijón.}}\footnote{Mecalux Software Solutions es la división de Mecalux dedicada enteramente al desarrollo de software para almacenes y logística.}

\begin{notebox}
  Aunque las prácticas fueron presenciales, Mecalux tiene una metodología de teletrabajo muy arraigada de la cual muchos empleados de IT siguen beneficiándose. \\ \\
  Es notable cómo esta flexibilidad está bien integrada en su rutina diaria, y parte de la comunicación con el equipo ha sido de de esta manera.
\end{notebox}

Durante estas prácticas de empresa, tuve la oportunidad de formar parte del equipo de Data Analytics en la división de MSS. En el próximo capítulo se detallarán las actividades llevadas a cabo por este equipo, junto con sus metas y objetivos.



%----------------------------------------------------------------------------------------
%	THESIS CONTENT - APPENDICES
%----------------------------------------------------------------------------------------

\appendix % Cue to tell LaTeX that the following "chapters" are Appendices

% Include the appendices of the thesis as separate files from the Appendices folder
% Uncomment the lines as you write the Appendices

%\include{Appendices/AppendixA}

%----------------------------------------------------------------------------------------
%	BIBLIOGRAPHY
%----------------------------------------------------------------------------------------

%\printbibliography[heading=bibintoc]
%\printbibliography

%----------------------------------------------------------------------------------------

\end{document}  
