% Chapter 3
\chapter{Lecciones aprendidas y conclusiones}
Durante el desarrollo de mis prácticas, he adquirido tanto habilidades blandas como habilidades técnicas. 

Nunca había utilizado \href{https://dotnet.microsoft.com/es-es/languages/csharp}{\bold{C\#}}, pero confiaba en la capacidad de abstraer mis conocimientos de otros lenguajes POO, y considero que no tengo nociones muy avanzadas en cuanto a inteligencia artificial respecta, pero si interés por la temática, ya que tenía pensado realizar por mi cuenta algo bastante parecido a lo que se requería en la oferta, por lo que me aventuré a hacerme cargo de las responsabilidades que se requerían para las prácticas, y estoy muy contento con la decisión.

He aprendido muchísimo más del campo de la inteligencia artificial, y he disfrutado mucho de toda la labor de investigación que realizamos para poder ofrecer soluciones al proyecto, conceptos como LoRA, RAG, o el descubrimiento de modelos \href{https://en.wikipedia.org/wiki/Large_language_model}{\bold{LLM}} emergentes que usan 2 bits para los pesos permitiendo ejecutar estos modelos en dispositivos de recursos limitados como podrían ser dispositivos móviles, entre otros muchos más conocimientos adquiridos.

He experimentado una gran satisfacción durante mis prácticas, ya que he tenido la oportunidad de trabajar con una variedad de herramientas, tecnologías y librerías (pgAdmin, \href{https://www.sqlite.org/}{\bold{SQLite}}, 
\href{https://es.wikipedia.org/wiki/CUDA}{\bold{CUDA}}, pgvector, llamasharp, llama.cpp), así como con diferentes lenguajes de programación (\href{https://en.wikipedia.org/wiki/C_(programming_language)}{\bold{C}}, \href{https://en.wikipedia.org/wiki/C%2B%2B}{\bold{C++}}, \href{https://dotnet.microsoft.com/es-es/languages/csharp}{\bold{C\#}}, \href{https://www.python.org/}{\bold{Python}}, Shell, PL/pgSQL) y sistemas operativos (Windows, WSL). Esta diversidad ha sido muy gratificante y ha complementado mi deseo de adquirir un conocimiento amplio y una visión general de múltiples áreas.

Los problemas que tuve con la librería de llamasharp me hizo crear mis primeras issues en GitHub lo cual junto a el uso de Git para el proyecto de MSSCopilot me acercó más a los conocimientos necesarios y buenas prácticas para el desarrollo de código colaborativo. 

La necesidad de comunicarme con mi compañero de prácticas de manera efectiva y aprovechar al máximo las virtudes de cada uno siendo consciente de nuestras flaquezas y diferencias me ha servido mucho como crecimiento personal a la hora de trabajar en equipo, un poco de liderazgo, y sobre todo de adaptabilidad y resolución de problemas.
\newpage
En general ha sido una experiencia en la que aunque la tutela del tutor ha sido excelente y ha estado presente y preocupado por nosotros en todo momento a pesar de sus responsabilidades, ha habido bastante autonomía y he sentido que se le ha dado mucha importancia a la capacidad de resolución de problemas y el pensamiento crítico lo cual valoro muchísimo.

Todo el departamento de Data Analytics me trató excelentemente y he tenido un trato muy cercano, en general con todo el personal de la empresa pero especialmente con ellos. Ha sido muy enriquecedor en todos los sentidos ver como opera una empresa de tal importancia como es \href{https://www.mecalux.es/}{\bold{Mecalux}}. 