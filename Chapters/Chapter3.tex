% Chapter 3
\chapter{Lecciones aprendidas y conclusiones}
Durante el desarrollo de mis prácticas, he adquirido y puesto en prueba tanto habilidades blandas como habilidades técnicas, alineadas con las competencias básicas, generales, y específicas de mi grado en Ingeniería Informática en Tecnologías de la Información.

Nunca había utilizado \href{https://dotnet.microsoft.com/es-es/languages/csharp}{\bold{C\#}}, pero confiaba en la capacidad de abstraer mis conocimientos de otros lenguajes POO, y considero que no tengo nociones muy avanzadas en cuanto a inteligencia artificial respecta, pero si interés por la temática, ya que tenía pensado realizar por mi cuenta algo bastante parecido a lo que se requería en la oferta, por lo que me aventuré a hacerme cargo de las responsabilidades que se requerían para las prácticas, y estoy muy contento con la decisión.

He aprendido muchísimo más del campo de la inteligencia artificial, y he disfrutado mucho de toda la labor de investigación que realizamos para poder ofrecer soluciones al proyecto, conceptos como LoRA, RAG, o el descubrimiento de modelos \href{https://en.wikipedia.org/wiki/Large_language_model}{\bold{LLM}} emergentes que usan 2 bits para los pesos permitiendo ejecutar estos modelos en dispositivos de recursos limitados como podrían ser dispositivos móviles, entre otros muchos más conocimientos adquiridos.

He experimentado una gran satisfacción durante mis prácticas, ya que he tenido la oportunidad de trabajar con una variedad de herramientas, tecnologías y librerías (pgAdmin, \href{https://www.sqlite.org/}{\bold{SQLite}}, 
\href{https://es.wikipedia.org/wiki/CUDA}{\bold{CUDA}}, pgvector, llamasharp, llama.cpp), así como con diferentes lenguajes de programación (\href{https://en.wikipedia.org/wiki/C_(programming_language)}{\bold{C}}, \href{https://en.wikipedia.org/wiki/C%2B%2B}{\bold{C++}}, \href{https://dotnet.microsoft.com/es-es/languages/csharp}{\bold{C\#}}, \href{https://www.python.org/}{\bold{Python}}, Shell, PL/pgSQL) y sistemas operativos (Windows, WSL). Esta diversidad ha sido muy gratificante y ha complementado mi deseo de adquirir un conocimiento amplio y una visión general de múltiples áreas.

Los problemas que tuve con la librería de llamasharp me hizo crear mis primeras issues en GitHub lo cual junto a el uso de Git para el proyecto de MSSCopilot me acercó más a los conocimientos necesarios y buenas prácticas para el desarrollo de código colaborativo. 
\newpage
La necesidad de comunicarme con mi compañero de prácticas de manera efectiva y aprovechar al máximo las virtudes de cada uno siendo consciente de nuestras flaquezas y diferencias me ha servido mucho como crecimiento personal a la hora de trabajar en equipo, un poco de liderazgo, y sobre todo de adaptabilidad y resolución de problemas.

En general ha sido una experiencia en la que aunque la tutela del tutor ha sido excelente y ha estado presente y preocupado por nosotros en todo momento a pesar de sus responsabilidades, ha habido bastante autonomía y he sentido que se le ha dado mucha importancia a la capacidad de resolución de problemas y el pensamiento crítico lo cual valoro muchísimo.

Todo el departamento de Data Analytics me trató excelentemente y he tenido un trato muy cercano, en general con todo el personal de la empresa pero especialmente con ellos. Ha sido muy enriquecedor en todos los sentidos ver como opera una empresa de tal importancia como es \href{https://www.mecalux.es/}{\bold{Mecalux}}. 

\section{Análisis de las características y perfil profesional del puesto desempeñado}

Durante mis prácticas en la empresa Mecalux, desempeñé el rol de desarrollador de software, con un enfoque particular en la inteligencia artificial y el desarrollo de aplicaciones. Estas se han alineado perfectamente con mi perfil profesional como futuro egresado de Ingeniería Informática, poniendo a prueba y desarrollando competencias que considero esenciales para mi carrera. Desde el inicio, se esperaba de mí un ojo crítico y habilidades avanzadas para resolver problemas, así como buenas habilidades de comunicación y pensamiento crítico, todas características indispensables para un ingeniero informático. A lo largo de mi formación y durante estas prácticas, he tenido la oportunidad de aplicar muchas de las competencias clave que se esperan de nosotros al finalizar el grado. 

A continuación, se presenta un análisis detallado de las características y el perfil profesional del puesto que desempeñé:

\subsection{Descripción del puesto}
El rol principal involucró el desarrollo de soluciones de inteligencia artificial utilizando diversas herramientas y tecnologías. Las responsabilidades específicas incluyeron la programación en lenguajes como \href{https://dotnet.microsoft.com/es-es/languages/csharp}{\bold{C\#}}, \href{https://www.python.org/}{\bold{Python}}, Shell, \href{https://en.wikipedia.org/wiki/C_(programming_language)}{\bold{C}} y \href{https://en.wikipedia.org/wiki/C%2B%2B}{\bold{C++}}, el uso de bases de datos como \href{https://www.sqlite.org/}{\bold{SQLite}} y \href{https://www.postgresql.org/}{\bold{PostgreSQL}}, y la implementación de algoritmos de inteligencia artificial.
\newpage
\subsection{Habilidades y competencias requeridas}
El puesto requirió una sólida comprensión de los principios de programación orientada a objetos, habilidades en el manejo de bases de datos, y conocimientos básicos en inteligencia artificial. Además, se necesitaba la capacidad de resolver problemas técnicos complejos y de trabajar de manera colaborativa en un equipo con cierta autonomía.

\subsection{Formación y conocimientos}
Mi formación en la carrera proporcionó una base sólida en programación y sistemas operativos, que fue crucial para cumplir con las responsabilidades del puesto. 

Sin embargo, la necesidad de aprender y adaptarme a nuevas tecnologías como \href{https://dotnet.microsoft.com/es-es/languages/csharp}{\bold{C\#}} y herramientas específicas de inteligencia artificial destacó la importancia de la capacidad de aprendizaje autónomo.

\subsection{Desarrollo profesional}
Estas prácticas me permitieron adquirir nuevas habilidades técnicas, como el uso de librerías de IA emergentes y una mejora sustancial en la gestión de proyectos de software colaborativo mediante Git. También mejoré mis competencias blandas, especialmente en comunicación efectiva y trabajo en equipo.

\subsection{Alineación con el perfil profesional}
El puesto se alinea estrechamente con mis objetivos profesionales de trabajar en el desarrollo de software y aplicaciones de inteligencia artificial. La experiencia práctica adquirida complementa mi formación académica y fortalece mi perfil profesional como futuro ingeniero informático.

\subsection{Evaluación personal}
Durante las prácticas, pude aplicar muchas de las competencias adquiridas en mi formación, tales como la capacidad de resolver problemas (GTR1), actuar autónomamente (GTR3), y comunicarme efectivamente (GTR6). Identifiqué áreas de mejora en el manejo avanzado de ciertas tecnologías y en la profundización de conocimientos en inteligencia artificial, que son objetivos para mi desarrollo futuro.

En resumen, las prácticas no solo me permitieron aplicar y expandir mis conocimientos técnicos, sino que también fueron fundamentales para mi crecimiento profesional y personal, alineándose perfectamente con los de la carrera.

\section{Sugerencias}
Aunque me gustaría ofrecer sugerencias constructivas para mejorar el proceso de prácticas, debo admitir que las experiencias vividas han sido increíblemente satisfactorias y considero que se han abordado de manera casi perfecta. Sin embargo, siempre hay aspectos que pueden ser optimizados para futuros alumnos.

Durante las prácticas, la comunicación se llevó a cabo principalmente a través de Teams, una plataforma que, si bien resultó familiar y eficiente debido a su uso constante durante mi formación académica, eché de menos la utilización de software especializado para empresas, como Jira o Slack. Estas herramientas, utilizadas por otros compañeros en sus prácticas, ofrecen funcionalidades específicas que podrían haber mejorado aún más la gestión y coordinación del proyecto. No obstante, entiendo que la elección de Teams se haya debido a su integración con la cuenta de Microsoft proporcionada al inicio de las prácticas, lo cual facilita la gestión de usuarios y recursos.

En cuanto a la experiencia personal, he notado que en algunas empresas, los alumnos reciben merchandising como mochilas, camisetas o termos. Esta práctica no solo contribuye a la promoción de la empresa, sino que también motiva a los alumnos al sentirse parte de la organización. Considero que proporcionar este tipo de obsequios podría fortalecer aún más el vínculo entre los estudiantes y Mecalux Software Solutions, y ayudar a difundir su nombre dentro del círculo académico de informática, lo cual sería beneficioso para ambas partes.