
% Chapter 1
\Chapter{Data Analytics}{Innovación y solución de problemas}

\section{Contexto}\label{sec:contexto}
He tenido la fortuna de trabajar dentro del departamento de Data Analytics, el cual se encarga de recopilar, limpiar e interpretar conjuntos de datos para responder a preguntas o resolver problemas. Es un equipo muy multidisciplinar, y me sorprendió positivamente que, aunque la mayoría de sus miembros no provienen de estudios de Informática (sino de áreas como Física, Matemáticas, etc.), todos poseen una gran capacidad para reflexionar y resolver problemas de manera lógica, metódica y, sobre todo, en equipo.

Además de su labor como analistas de datos, este equipo también cumple la función de resolver inconvenientes que puedan surgir en otros departamentos y se encarga de investigar y poner a prueba nuevas soluciones antes de su implementación en el entorno de producción. Se podría decir que realizan funciones similares a las de I+D+I, ya que investigan, desarrollan y prueban nuevas soluciones para mejorar los procesos y resolver problemas dentro de la empresa.

Durante mis prácticas, mi labor ha sido una combinación de investigación y desarrollo de software, lo cual ha sido posible gracias a la naturaleza y dinámica de este equipo.


\section{Metodología de Trabajo}
Como se mencionó en la introducción, parte del equipo realiza teletrabajo, algunos de manera ocasional y otros de forma regular. De hecho, hay un integrante que ni siquiera reside en Asturias. El equipo realiza reuniones diarias (dailys), a las cuales tuve el placer de asistir en la última etapa de mis prácticas. En estas reuniones, los miembros explican el progreso y las tareas realizadas el día anterior. Estas reuniones se llevan a cabo en salas especialmente equipadas en el edificio, las cuales cuentan con cámaras y proyectores para que los empleados que están teletrabajando puedan participar activamente en las reuniones.

\subsection{Dinámica de trabajo durante las prácticas}
El departamento estaba enfocado en sus proyectos, por lo que mi equipo de trabajo habitual se 
redujo a otro alumno de prácticas de la carrera de Ingeniería Informática del Software, y nuestro tutor, 
perteneciente a Data Analytics, que se encargaba de que fuésemos por el buen camino y con el que hacíamos
también nuestras propias reuniones diarias para ver el progreso de nuestro trabajo.

Mi compañero y yo teníamos horarios diferentes, por lo que coincidíamos en las oficinas durante un tiempo limitado, lo que conllevó a la necesidad de coordinarnos lo mejor posible y así poder tener los objetivos claros y aprovechar al máximo el tiempo que teníamos juntos para desarrollar, estructurar nuestras tareas y debatir los problemas a los que nos enfrentaríamos. 

Al entrar antes a la oficina, me encargaba de hablar con el tutor para ver si había nuevas opciones que explorar o investigar, por donde seguir si habíamos conseguido los resultados esperados, o en caso contrario comunicar los inconvenientes, y coordinar la planificación para el día. 

De esta manera, cuando mi compañero llegaba, era todo más fácil para los tres, el tutor no necesitaba volver a explicarlo todo y yo podía comunicarle de forma efectiva que teníamos que hacer, las novedades, y como nos podríamos distribuir el trabajo. 

Por otra parte, cuando yo me marchaba mi compañero seguía desarrollando y trabajando en el proyecto, por lo que me comunicaba sus avances por la plataforma de Teams para que cuando llegase al día siguiente supiese de manera más rápida que era lo que se había hecho.

Nuestra comunicación fue en todo momento muy buena, nos sentábamos en mesas próximas por lo que nos acercábamos a hablar sobre como realizar el trabajo frecuentemente, y cuando no queríamos molestar utilizábamos el chat de Teams.

Cabe resaltar que aunque no trabajase con todos los integrantes del departamento de Data Analytics directamente al no pertenecer al proyecto en sí, muchos de ellos me ayudaron  a lo largo de las prácticas y hubo una comunicación cercana y efectiva independientemente de ello. 