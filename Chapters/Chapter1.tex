
% Chapter 1
\Chapter{Introducción 2}{El alumno y la empresa}

\section{La Empresa Mecalux}
\href{https://www.mecalux.es/}{\textbf{Mecalux}} es una empresa reconocida internacionalmente en el sector de soluciones de almacenamiento. Desde su fundación en 1966, ha desarrollado un amplio portafolio que abarca una variedad de sistemas de almacenamiento, como estanterías, almacenes automatizados y soluciones de software para la gestión de almacenes.

\section{Las Prácticas Realizadas}
Con una presencia global, Mecalux opera en numerosos países, gestionando un gran volumen de operaciones y personal especializado. Las prácticas fueron realizadas en las \href{https://maps.app.goo.gl/bJKvSNHAo5t1j4BZ6}{\textbf{oficinas de MSS}}\footnote{Mecalux Software Solutions es la división de Mecalux dedicada enteramente al desarrollo de software para almacenes y logística.} en Gijón.

Durante estas prácticas de empresa, tuve la oportunidad de formar parte del equipo de Data Analytics en el departamento de MSS. En el próximo capítulo se detallarán las actividades llevadas a cabo por este equipo, junto con sus metas y objetivos.

\section{Metodología de Trabajo}
Aunque las prácticas fueron presenciales, Mecalux tiene una metodología de teletrabajo muy arraigada de la cual muchos empleados de IT siguen beneficiándose. Es notable cómo esta flexibilidad está bien integrada en su rutina diaria.

\section{Estructura del Documento}
Esta tesis se organiza de la siguiente manera:
\begin{itemize}
    \item \textbf{Capítulo 2}: Revisión de la literatura.
    \item \textbf{Capítulo 3}: Descripción de la metodología utilizada.
    \item \textbf{Capítulo 4}: Presentación de los resultados.
    \item \textbf{Capítulo 5}: Conclusiones finales y posibles trabajos futuros.
\end{itemize}

\chapter{Revisión de la Literatura}

\section{Estudios Previos}
En esta sección se revisarán los estudios previos relacionados con el tema de la tesis.\\
\\
El análisis de la literatura existente es crucial para establecer un marco teórico sólido.

\subsection{Investigaciones Relacionadas}
Se analizarán investigaciones relevantes realizadas en los últimos años.\\
\\
Esto incluye estudios sobre la gestión de almacenes y las soluciones de software aplicadas.

\section{Teorías y Modelos}
Además, se discutirán las principales teorías y modelos que sustentan el campo de estudio.

\subsection{Modelos de Almacenamiento}
Se presentarán modelos teóricos que explican el comportamiento y la eficiencia de los sistemas de almacenamiento.

\subsection{Teorías de Gestión}
Las teorías de gestión que influyen en la implementación de sistemas automatizados en almacenes también serán revisadas.

